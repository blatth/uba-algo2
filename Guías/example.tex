\documentclass{article}
\usepackage{amsmath, amssymb}

\begin{document}

\title{Especificación de problemas}
\date{}
\author{Blatth}
\maketitle


\section{Funciones auxiliares}

\paragraph{2.b}
\texttt{pred mayorPrimoQueDivide} (x, y: $\mathbb{Z}$) \{
\[
\qquad (\forall p : \mathbb{Z}) ((esPrimo(p) \land_{L} x \mod p = 0) \rightarrow_{L} p \leq y) \land esPrimo(y) \land_{L} x \mod y = 0 
\]
\qquad \} 
\bigbreak
Acá estoy definiendo que existe un p \textbf{menor o igual} a y que divide a x. Luego también defino que y \textbf{es primo} y divide a x.

\paragraph{4.c}
\texttt{pred hayUnoParQueDivideAlResto} (s: \textit{seq} $\langle \mathbb{Z} \rangle$) \{
\[
\qquad (\exists n : \mathbb{Z})((n \in s \land esPar(n)) \land (\forall m : \mathbb{Z})(m \in s \rightarrow n \neq 0 \land_{L} m \mod n = 0))
\]
\qquad \}
\bigbreak
Acá planteo dos cosas: que n esté contenido en s y a su vez sea par. Además de eso, quiero que n $\neq$ 0 (para que no se indefina) y que m (que sería una representación de todos los números contenidos en s, por eso se utiliza $\forall$m:$\mathbb{Z}$) sea divisible por él.
\bigbreak
Donde \textbf{esPar} está definida como:
\bigbreak
\texttt{pred esPar (n:$\mathbb{Z}$)} \{
\[
n \mod 2 = 0
\]
\qquad \}


\paragraph{4. d}
\texttt{pred enTresPares} (s: \textit{seq} $\langle \mathbb{Z} \rangle$) \{
\[
estaOrdenada(s) \land ( \forall i : \mathbb{Z})(0 \leq i < |s|) \rightarrow_{L} s[i] = 0 \vee s[i] = 1 \vee s[i] = 2
\]
\qquad \}
\bigbreak
En este caso, llamo al auxiliar estáOrdenada que se asegura que cada elemento sea menor o igual a su siguiente. Luego planteo que $\forall$ i $\in \mathbb{Z}$ los valores solamente puedan ser 0, 1 ó 2.
\bigbreak
Donde \textbf{estáOrdenada} es definida como:
\bigbreak
\texttt{pred estáOrdenada} (s: \textit{seq} $\langle \mathbb{Z} \rangle$) \{
\[
(\forall i : \mathbb{Z})(0 \leq i \leq |s|-2) \rightarrow_{L} s[i] \leq s[i+1]
\]
\qquad \}
\texttt{}
\end{document}